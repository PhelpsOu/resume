% !TEX TS-program = xelatex
% !TEX encoding = UTF-8 Unicode
% !Mode:: "TeX:UTF-8"

\documentclass{resume}
\usepackage{zh_CN-Adobefonts_external} % Simplified Chinese Support using external fonts (./fonts/zh_CN-Adobe/)
\usepackage{linespacing_fix} % disable extra space before next section
\usepackage{cite}

\begin{document}
\pagenumbering{gobble} % suppress displaying page number

\name{欧岳枫}

\basicInfo{
  \email{ouyuefeng@sina.com} \textperiodcentered\ 
  \phone{(+86) 130-236-78176} \textperiodcentered\ 
  \linkedin[ouyuefeng]{https://www.linkedin.com/in/ou-yuefeng-94316b162/}}
 
\section{\faGraduationCap\  教育背景}
\datedsubsection{\textbf{浙江大学}, 杭州}{2019 -- 2022}
\textit{硕士研究生}\ 电子与通信工程
\datedsubsection{\textbf{浙江大学}, 杭州}{2015 -- 2019}
\textit{学士}\ 电子科学与技术

\section{\faUsers\ 工作经历}
\datedsubsection{\textbf{腾讯集团-微信事业群-微信支付线-商业支付交易订单组}}{2023年3月 -- 至今}
\role{C++, Python 服务端开发}{深圳}
\begin{itemize}
  \item \textbf{核心交易订单可用性建设}:以多园区条带化部署和无缝跳单设计为核心,辅以资源拆分、依赖整治等手段,共同保障日均十数亿交易量级的商业支付核心订单层可用性;
  \item \textbf{离线商户账单数据交付重构}:在领域驱动设计思想的指导下,通过深度领域建模,设计并实现了异构多路商户账单离线数据交付系统,提升了可读性、可运营性和可扩展性,降低交付新数据需求所需的开发量,为百亿级资金出账系统的可用性提供保障;
  \item \textbf{脚本机上云}:实现了将订单离线脚本任务从物理机到云 CVM 机器的迁移,将原有的实现从 Python2 升级到 Python3,将本地日志升级为远程日志,提升了订单离线任务(内部对账、外部对账、异常订单调账)的可用性、可运营性和可维护性;
  \item \textbf{数据仓库建设}:梳理订单数据血缘,盘点数据资产和扇出,逐步回收和治理不符合数据仓库规范的数据表,将 ODS 表收敛至组内,逐步规范各需求方的数据消费,以定制化的维表进行交付;
  \item \textbf{新支付方式接入支持}:承接支付产品合规和创新需求,在订单层支持了新的支付方式,适配新支付渠道的事务资源,改进异常关单的分布式事务流程,开发适配新支付方式的集成测试用例。
\end{itemize}

\datedsubsection{\textbf{腾讯集团-微信事业群-微信支付线-社交支付组}}{2022年4月 -- 2023年2月}
\role{C++ 服务端开发}{深圳}
\begin{itemize}
  \item \textbf{交付创新产品价值}:参与交付微信支付手表端 APP、儿童零花钱支付、微信支付接入数字人民币、转账入经营账户等创新支付产品,持续为微信支付用户提供安全、稳定、可靠、创新的支付体验;
  \item \textbf{适应组织需求的全栈化能力}:作为服务端开发,主动拓宽技术栈,学习 NodeJS/TS 等前端知识,完成数字人民币钱包快付签约仿原生小程序页面开发,设计并完成内部运营系统权限管控 web 页面的前后端开发,为组织交付价值;
  \item \textbf{保障产品正确性和可运营性的数据工程}:实现支付产品交易数据审计,保障交易安全性和正确性,提升用户体验;通过数据聚合、清洗等手段,交付多样化的数据分析报表,为产品运营和策略制定提供真实准确的参考数据。
\end{itemize}

\datedsubsection{\textbf{(实习)阿里巴巴集团-新零售事业群-CBU 技术部-搜索工程组}}{2021年5月 -- 2021年8月}
\role{C++ 后端开发实习生}{杭州}
\begin{itemize}
  \item \textbf{搜索意图分析模块}:搜索意图分析模块提供 Query 分词、Query 纠错/改写/扩展、权值计算、类目预测、属性排序、智能导航以及下拉推荐等功能,是 1688 主搜的重要入口模块;实习过程中主要负责将运行在物理机上的旧版代码迁移至集团中间件,部分重构代码以适应新平台,提升模块可用性,减少冗余代码。
\end{itemize}

\section{\faCogs\ 技能}
% increase linespacing [parsep=0.5ex]
\begin{itemize}[parsep=0.5ex]
  \item \textbf{编程语言}: C++ / Python
  \item \textbf{软件工程}: 需求分析,领域建模,序列图,领域驱动设计,重构已有设计
  \item \textbf{综合能力}: 结构化思维,契约化开发,防御性编程,面向异常、可用性、可扩展性、可测试性设计
\end{itemize}

% \section{\faInfo\ 其他}
% \begin{itemize}[parsep=0.5ex]
%   \item 语言: 良好的英语口语、英文语料阅读能力
% \end{itemize}
\end{document}
